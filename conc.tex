\section{Conclusions}
\label{sec:conc}

An attacker can implement a DNS rebinding attack to circumvent firewalls and confuse the browser into breaking the same-origin policy. 
While many existing approaches towards exploiting DNS rebinding vulnerabilities have been fixed, many new vulnerabilities are still being discovered. Existing defenses attempt protect against specific attack vectors, but do not prevent DNS rebinding attacks as a whole. These attacks are highly cost effective, relatively quick to execute, and are capable of doing severe damage to both the victim and the intranet to which he is connected. The ability to interactively communicate with the otherwise inaccessible server gives the attacker even more power. The attacker can hole pucnhing into firewall, scan internal networks, access and infiltrate private nodes on the network, uncover sensitive information, modify the state of web pages under the IP address of the victim, login and authenticate as another user, and hijack the victim's IP address for use in a botnet.

DNS rebinding attacks have been around for more than 15 years, many defenses have been presented in previous work for preventing traditional DNS rebinding attacks but the threat hasn't been completely removed. We present possible defenses against the DNS cache flooding technique we introduced in this paper. Increasing the cache size can help make the attack prohibitively impractical to execute, while smarter cache eviction could potentially eliminate this particular form of DNS rebinding altogether. We believe that DNS rebinding is still a very important and dangerous exploit, and hope that future work in this area will explore new vulnerabilities.
