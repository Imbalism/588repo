\section{Evaluation}
\label{sec:eval}

We developed a proof-of-concept exploits for DNS flooding based rebinding vulnerabilities using pure Javascript and HTML5. Our system consists of a custom DNS server authoritative for takenoteswith.us.

We measured our DNS rebinding attack by two primary factors. We analyzed the \emph{time to launch} of the attack and the \emph{impact} of the attack. We compared it two DNS rebinding mechanisms we've implemented. 

\subsection{Time to Launch}
To protect from a time-varying attack that is able to establish an interactive session between the malicious server and the broswer, most modern browers have implemented DNS pinning that pin a DNS record for a period of time. At this time, a time-varying attack would take ~160 seconds to launch according to our experiment on the latest Chrome browser. However, by flooding the DNS table, we found that in the current Chrome implementation, if a DNS record is kicked out from the cache, the pinning time would dramatically decrease. We found that in our attack, only ~10 seconds is needed to launch the attack on a browser that have 100 entries. 

In early 2013, the Chromium community has increased the size of DNS cache from 100 to 1000. This is not a really security patch but a performance related patch. We then ran our experiments on the staging version Chrome, and found that it would only take 10 more seconds to flood the DNS table and launch the attack. 

Another approach of DNS rebinding is based on multiple A records attack. This attack needs only a small amount of time due to the number of packets transmitted, however this attack has certain limitations, we will discuss it in the next subsection.

\subsection{Impact}
We now evaluate the impact of our attack against other DNS rebinding approaches. As mentioned in the last section, the multiple DNS record approach has the advantage of using less time to launch. However, it has several limitations on its impacts. 1) The rebinded IP address cannot be an internal IP address, otherwise the browser will priorize it and select it in the first place which results in a failure in DNS rebinding. 2) The attacker cannot change the rebinded IP address on the fly, which makes it unable to scan the subnet. 

For time-varying attack, although it is possible to bind to an internal IP address, it is also hard to change the rebinded IP address on the fly due to the extremly long launching time. 

In our experiement, we are able to use FireDrill to rebind the domain name to an internal IP address to build a interactive session. Also, we are able to dynamically change the IP address during an attack. The attacker has the ability to navigate through the entire subnet instead of just one single IP address.
